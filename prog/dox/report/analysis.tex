\section{Analysis}
\subsection{DC Armature Model}
A DC motor can be described by two equations:
\begin{equation}
\label{eq:electrical}
	V(t) = L \frac{di(t)}{dt} + Ri(t) + E(t)
\end{equation}
\begin{equation}
\label{eq:mechanical}
	J \dot{\omega}(t) = k_{tau} i(t) - \mu \omega(t) - T_{L}(t),
\end{equation}
where \ref{eq:electrical} describes electrical and \ref{eq:mechanical} describes mechanical aspect of the motor.

It is known that
\begin{equation}
\label{eq:back_emf}
	E(t) = k_{emf} \omega(t),
\end{equation}
\begin{equation}
\label{eq:curr_force}
	T(t) = k_{tau} i(t).
\end{equation}

Combining the above equations and assuming that changes in current are instantaneous yields
\begin{equation}
\label{eq:feedforward_trans}
	V(t) = \frac{R}{k_{tau}} (J \dot{\omega}(t) + \mu \omega(t) + T_{L}(t) ) + k_{emf} \omega(t)
\end{equation}
or simply
\begin{equation}
\label{eq:feedforward_trans_reduced}
	V(t) = Ri(t) + k_{emf} \omega(t)
\end{equation}
if the current is measured. 

Equations \ref{eq:feedforward_trans} and \ref{eq:feedforward_trans_reduced} hold for every state of the motor. However, since we want the gripper to hold an object firmly, desired $\omega$ and $\dot{\omega}$ are zero. That means the final feedforward equation when holding an object can be reduced to
\begin{equation}
\label{eq:feedforward_reduced}
	V(t) = R \frac{T_{L}(t)}{k_{tau}} = R i(t).
\end{equation}

Nomenclature:
$R$ - motor terminal resistance,
$L$ - motor terminal inductance,
$i$ - current,
$E$ - back electromotive force,
$J$ - rotor inertia moment,
$\omega$ - rotor angular velocity,
$\dot{\omega}$ - rotor angular acceleration,
$k_{tau}$ - torque constant,
$k_{emf}$ - speed constant,
$T_{L}$ - load torque. 